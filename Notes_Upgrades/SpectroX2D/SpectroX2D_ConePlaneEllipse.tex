\documentclass[a4paper,11pt,twoside,titlepage,openright]{book}

\usepackage[english]{babel}
\usepackage{color}
\usepackage{graphicx}
\usepackage{amsmath}
\numberwithin{equation}{section}
\usepackage[margin=3cm]{geometry}
\usepackage{hyperref}
\usepackage{epsfig,amsfonts}
\usepackage{xcolor,import}


\pagestyle{plain}

\newcommand{\ud}[1]{\underline{#1}}
\newcommand{\lt}{\left}
\newcommand{\rt}{\right}
\DeclareMathOperator{\e0}{\epsilon_0}
\DeclareMathOperator{\n}{\underline{n}}
\DeclareMathOperator{\ei}{\underline{e}_1}
\DeclareMathOperator{\et}{\underline{e}_2}
\DeclareMathOperator{\ex}{\underline{e}_x}
\DeclareMathOperator{\ey}{\underline{e}_y}
\DeclareMathOperator{\ez}{\underline{e}_z}
\DeclareMathOperator{\DD}{\cos(\theta)^2 - \sin(\psi)^2}
\newcommand{\wdg}{\wedge}
\newcommand{\hypot}[1]{\textbf{\textcolor{green}{#1}}}


\begin{document}

\title{ToFu geometric tools\\ Intersection of a cone with a plane}
\author{Didier VEZINET}
\date{15.10.2019}
\maketitle

\tableofcontents

\chapter{Geometry}
\label{chap:Definitions}

\section{Generic cone and plane}

Let's consider a half-cone $C_1$ (defined only for $z > 0$), with summit on the cartesian frame's origin $(O, \ex, \ey, \ez)$.
The cone's axis is the $(O,\ez)$ axis.
It's angular opening is $\theta$.

Let's consider plane $P_1$, of normal $\n$, intersection axis $(O,\ez)$ at point $P$ of coordinates $(0,0,Z_P)$.
Vector $\n$ is oriented by angles $\phi$ and $\psi$ such that one can define the local frame $(P, \ei, \et, \n)$:
$$
\lt\{
	\begin{array}{ll}
		\ei & = \cos(\phi)\ex + \sin(\phi)\ey\\
		\et & = \lt(-\sin(\phi)\ex + \cos(\phi)\ey\rt)\cos(\psi) + \sin(\psi)\ez\\
		\n & = \ei \wdg \et\\
		   & = \lt( \sin(\phi)\ex - \cos(\phi)\ey \rt)\sin(\psi) + \cos(\psi)\ez
	\end{array}
\rt.
$$

We want to find all points $M$ of coordinates $(x, y, z)$ and $(x_1, x_2)$ belonging both to the cone $C_1$ and the plane $P_1$.

$$
M \in C_1 \Leftrightarrow \underline{OM}.\ez = \cos(\theta) \|\underline{OM}\|
$$

$$
M \in P_1 \Leftrightarrow \underline{PM}.\n = 0
$$


\section{Intersection}

If $M$ belongs to both $P_1$ and $C_1$, then:
$$
(\underline{OM}.\ez)^2 = \cos(\theta)^2 \|\underline{OM}\|^2
$$

Given that:
$$
\begin{array}{lll}
	\underline{OM} & = \underline{OP} + \underline{PM}\\
		       & = Z_P\ez + x_1\ei + x_2\et\\
		       & = Z_P\ez + x_1\lt(\cos(\phi)\ex + \sin(\phi)\ey\rt) + x_2\lt(\lt(-\sin(\phi)\ex + \cos(\phi)\ey\rt)\cos(\psi) + \sin(\psi)\ez\rt)\\
		       & = Z_P\ez + x_1\cos(\phi)\ex + x_1\sin(\phi)\ey - x_2\sin(\phi)\cos(\psi)\ex + x_2\cos(\phi)\cos(\psi)\ey + x_2\sin(\psi)\ez\\
	               & = \lt(x_1\cos(\phi) - x_2\sin(\phi)\cos(\psi)\rt)\ex + \lt(x_1\sin(\phi) + x_2\cos(\phi)\cos(\psi)\rt)\ey + \lt(Z_P + x_2\sin(\psi)\rt)\ez\\
\end{array}
$$

We have:
$$
\begin{array}{lll}
	(\underline{OM}.\ez)^2  & = \lt(Z_P + x_2\sin(\psi)\rt)^2\\
				& = Z_P^2 + 2Z_Px_2\sin(\psi) + x_2^2\sin(\psi)^2
\end{array}
$$

And:
$$
\begin{array}{lll}
	\|\underline{OM}\|^2 	& = & \| \lt(x_1\cos(\phi) - x_2\sin(\phi)\cos(\psi)\rt)\ex + \lt(x_1\sin(\phi) + x_2\cos(\phi)\cos(\psi)\rt)\ey + \lt(Z_P + x_2\sin(\psi)\rt)\ez\|^2\\
				& = & \lt( x_1\cos(\phi) - x_2\sin(\phi)\cos(\psi) \rt)^2\\
				&   & + \lt(x_1\sin(\phi) + x_2\cos(\phi)\cos(\psi)\rt)^2\\
				&   & + \lt(Z_P + x_2\sin(\psi)\rt)^2\\
				& = & x_1^2\cos(\phi)^2 - 2x_1x_2\cos(\phi)\sin(\phi)\cos(\psi) + x_2^2\sin(\phi)^2\cos(\psi)^2\\
				&   & + x_1^2\sin(\phi)^2 + 2x_1x_2\sin(\phi)\cos(\phi)\cos(\psi) + x_2^2\cos(\phi)^2\cos(\psi)^2\\
				&   & + Z_P^2  + 2Z_Px_2\sin(\psi) + x_2^2\sin(\psi)^2\\
				& = & x_1^2 + x_2^2\cos(\psi)^2\\
				&   & + Z_P^2  + 2Z_Px_2\sin(\psi) + x_2^2\sin(\psi)^2\\
				& = & x_1^2 + x_2^2 + 2Z_Px_2\sin(\psi) + Z_P^2\\
\end{array}
$$

Thus:
$$
\begin{array}{lll}
	& (\underline{OM}.\ez)^2 = \cos(\theta)^2 \|\underline{OM}\|^2\\
	\Leftrightarrow & Z_P^2 + 2Z_Px_2\sin(\psi) + x_2^2\sin(\psi)^2 = \cos(\theta)^2 \lt(x_1^2 + x_2^2 + 2Z_Px_2\sin(\psi) + Z_P^2\rt)\\
	\Leftrightarrow & Z_P^2\lt(1-\cos(\theta)^2\rt) + 2Z_Px_2\sin(\psi)\lt(1-\cos(\theta)^2\rt) = x_1^2\cos(\theta)^2 + x_2^2\lt(\cos(\theta)^2 - \sin(\psi)^2\rt)\\
	\Leftrightarrow & Z_P^2\sin(\theta)^2 + 2Z_Px_2\sin(\psi)\sin(\theta)^2 = x_1^2\cos(\theta)^2 + x_2^2\lt(\cos(\theta)^2 - \sin(\psi)^2\rt)\\
\end{array}
$$

Considering that by hypothesis $\theta>0$:
$$
\begin{array}{lll}
	& (\underline{OM}.\ez)^2 = \cos(\theta)^2 \|\underline{OM}\|^2\\
	\Leftrightarrow & x_1^2\cos(\theta)^2 + x_2^2\lt(\DD\rt) - 2Z_Px_2\sin(\psi)\sin(\theta)^2 - Z_P^2\sin(\theta)^2 = 0\\
	\Leftrightarrow & x_1^2\frac{\cos(\theta)^2}{\DD} + x_2^2 - 2x_2Z_P\frac{\sin(\psi)\sin(\theta)^2}{\DD} - Z_P^2\frac{\sin(\theta)^2}{\DD} = 0\\
	\Leftrightarrow & x_1^2\frac{\cos(\theta)^2}{\DD} + \lt(x_2 - Z_P\frac{\sin(\psi)\sin(\theta)^2}{\DD}\rt)^2 - Z_P^2\frac{\sin(\psi)^2\sin(\theta)^4}{\lt(\DD\rt)^2} - Z_P^2\frac{\sin(\theta)^2}{\DD} = 0\\
	\Leftrightarrow & x_1^2\frac{\cos(\theta)^2}{\DD} + \lt(x_2 - Z_P\frac{\sin(\psi)\sin(\theta)^2}{\DD}\rt)^2 = Z_P^2 \frac{\sin(\theta)^2}{\lt(\DD\rt)^2} \lt(\sin(\psi)^2\sin(\theta)^2 + \DD \rt)\\
	\Leftrightarrow & x_1^2\frac{\cos(\theta)^2}{\DD} + \lt(x_2 - Z_P\frac{\sin(\psi)\sin(\theta)^2}{\DD}\rt)^2 = Z_P^2 \frac{\sin(\theta)^2}{\lt(\DD\rt)^2} \lt(-\sin(\psi)^2\cos(\theta)^2 + \cos(\theta)^2 \rt)\\
	\Leftrightarrow & x_1^2\frac{\cos(\theta)^2}{\DD} + \lt(x_2 - Z_P\frac{\sin(\psi)\sin(\theta)^2}{\DD}\rt)^2 = Z_P^2 \frac{\sin(\theta)^2\cos(\psi)^2\cos(\theta)^2}{\lt(\DD\rt)^2} \\
	\Leftrightarrow & \frac{x_1^2}{ Z_P^2\frac{\sin(\theta)^2\cos(\psi)^2}{\DD} } + \frac{\lt(x_2-Z_P\frac{\sin(\psi)\sin(\theta)^2}{\DD}\rt)^2}{ Z_P^2\frac{\sin(\theta)^2\cos(\psi)^2\cos(\theta)^2}{\lt(\DD\rt)^2} } = 1
\end{array}
$$

Or, in a reduced conic form:

$$
\frac{x_1^2}{ a^2 } + \frac{\lt(x_2-x_2(C)\rt)^2}{b^2} = 1
$$


With:
$$
\lt\{
	\begin{array}{lll}
		x_2(C) & = Z_P\frac{\sin(\psi)\sin(\theta)^2}{\DD} & \text{    $x_2$ coordinate of the center} \\
		a^2 & = Z_P^2\frac{\sin(\theta)^2\cos(\psi)^2}{\DD} & \text{    squared minor radius} \\
		b^2 & = Z_P^2\frac{\sin(\theta)^2\cos(\psi)^2\cos(\theta)^2}{\lt(\DD\rt)^2} & \text{    squared major radius} \\
		b^2 & = a^2\frac{\cos(\theta)^2}{\DD} \Leftrightarrow a^2 = b^2\lt(1-\frac{\sin(\psi)^2}{\cos(\theta)^2}\rt)
	\end{array}
\rt.
$$

The distance $d_{CF}$ between the center $C$ and the focal point $F$ can be deduced from:
$$
\begin{array}{lll}
	d_{CF}^2 & = b^2-a^2\\
		& = b^2\frac{\sin(\psi)^2}{\cos(\theta)^2}\\
		& = Z_P^2\frac{\sin(\theta)^2\cos(\psi)^2\sin(\psi)^2}{\lt(\DD\rt)^2}
\end{array}
$$

Hence, the $x_2$ coordinate of $F$ is:
$$
\begin{array}{lll}
	x_2(F) & = x_2(C) \pm d_{CF}\\
	       & = Z_P\frac{\sin(\psi)\sin(\theta)^2}{\DD} \pm Z_P\frac{\sin(\theta)\cos(\psi)\sin(\psi)}{\DD}\\
	       & = Z_P\frac{\sin(\psi)\sin(\theta)^2 \pm \sin(\theta)\cos(\psi)\sin(\psi)}{\DD}\\
	       & = Z_P\frac{\sin(\psi)\sin(\theta)}{\DD}\lt(\sin(\theta) \pm \cos(\psi)\rt)\\
\end{array}
$$

It is worth noticing that the neither the focal point nor the center correspond to the intersection between the axes and the plane $P$.


\section{Parametric equation}

In our case, only the axes $(O, \ez)$, fixed by the crystal's summit and normal, is independent from the cone's angular opening $\theta$.
It makes sense to design an ad-hoc coordinate system centered on the ellipse's center $C$ to use its parameterized equation.

Knowing all geometrical parameters, it is possible to compute all points on the ellipse parameterizing them with angle $\epsilon$:
$$
\lt\{
	\begin{array}{lll}
		x_1 = a\cos(\epsilon)\\
		x_2 = x_2(C) + b\sin(\epsilon)
	\end{array}
\rt.
$$

\subsection{From bragg angle and to local cartesian coordinates}

Keep in mind that the frame $(P, \ei, \et)$ is, by definition aligned on the minor and major axes of the ellipse.
Hence, for an arbitrary frame $(R, \ud{e}_i, \ud{e}_j)$ on plane $P_1$, translated and rotated by $\alpha$ with respect to $(P, \ei, \et)$:

$$
\lt\{
	\begin{array}{lll}
		\ud{e}_i = \cos(\alpha)\ei + \sin(\alpha)\et\\
		\ud{e}_j = -\sin(\alpha)\ei + \cos(\alpha)\et\\
		\ei = \cos(\alpha)\ud{e_i} - \sin(\alpha)\ud{e}_j
		\et = \sin(\alpha)\ud{e_i} + \cos(\alpha)\ud{e}_j
	\end{array}
\rt.
$$

Or, in coordinate tranforms:
$$
\lt\{
	\begin{array}{lll}
		x_1 = x_1(R) + x_i\cos(\alpha) - x_j\sin(\alpha)\\
		x_2 = x_2(R) + x_i\sin(\alpha) + x_j\cos(\alpha)\\
		x_i = (x_1-x_1(R))\cos(\alpha) + (x_2-x_2(R))\sin(\alpha)\\
		x_j = -(x_1-x_1(R))\sin(\alpha) + (x_2-x_2(R))\cos(\alpha)
	\end{array}
\rt.
$$

Hence:
$$
\lt\{
	\begin{array}{lll}
		x_i = \lt(a\cos(\epsilon)-x_1(R)\rt)\cos(\alpha) + \lt(x_2(C)-x_2(R) + b\sin(\epsilon)\rt)\sin(\alpha)\\
		x_j = -\lt(a\cos(\epsilon)-x_1(R)\rt)\sin(\alpha) + \lt(x_2(C)-x_2(R) + b\sin(\epsilon)\rt)\cos(\alpha)
	\end{array}
\rt.
$$


\subsection{From local cartesian coordinates to bragg angle}

Knowing $(x_i, x_j)$ and all geometric parameters, we now want to derive $(\theta, \epsilon)$.

From the previous equation, we can write:
$$
\lt\{
	\begin{array}{lll}
        x_i\cos(\alpha) - x_j\sin(\alpha) = a\cos(\epsilon)-x_1(R) & (1)\\
        x_i\sin(\alpha) + x_j\cos(\alpha) = x_2(C)-x_2(R) + b\sin(\epsilon) &
        (2)
	\end{array}
\rt.
$$

The dependency in $\theta$ is hidden in the expressions of $a$, $b$ and
$x_2(C)$.

By squaring and summing, it is possible to get rid of the $\epsilon$
dependency:
$$
\lt\{
	\begin{array}{lll}
        a^2\cos(\epsilon)^2 = \lt(x_i\cos(\alpha) - x_j\sin(\alpha) +
        x_1(R)\rt)^2\\
        b^2\sin(\epsilon)^2 = \lt(x_i\sin(\alpha) + x_j\cos(\alpha) - x_2(C) +
        x_2(R)\rt)^2
	\end{array}
\rt.
$$

Hence, keeping in mind that $a^2 = b^2\frac{\DD}{\cos(\theta)^2}$:
$$
\begin{array}{lll}
    & b^2 = \frac{\cos(\theta)^2}{\DD}\lt(x_i\cos(\alpha) - x_j\sin(\alpha) +
    x_1(R)\rt)^2 + \lt(x_i\sin(\alpha) + x_j\cos(\alpha) - x_2(C) +
    x_2(R)\rt)^2\\
    \Leftrightarrow &
    b^2\DD = \cos(\theta)^2\lt(x_i\cos(\alpha) - x_j\sin(\alpha) +
    x_1(R)\rt)^2 + \lt(\DD\rt)\lt(x_i\sin(\alpha) + x_j\cos(\alpha) - x_2(C) +
    x_2(R)\rt)^2\\
    \Leftrightarrow &
    Z_P^2\frac{\sin(\theta)^2\cos(\psi)^2\cos(\theta)^2}{\DD} = \cos(\theta)^2\lt(x_i\cos(\alpha) - x_j\sin(\alpha) +
    x_1(R)\rt)^2 + \lt(\DD\rt)\lt(x_i\sin(\alpha) + x_j\cos(\alpha) - x_2(C) +
    x_2(R)\rt)^2\\
    \Leftrightarrow &
    Z_P^2\sin(\theta)^2\cos(\psi)^2\cos(\theta)^2 = \cos(\theta)^2\lt\DD\rt)\lt(x_i\cos(\alpha) - x_j\sin(\alpha) +
    x_1(R)\rt)^2 + \lt(\DD\rt)^2\lt(x_i\sin(\alpha) + x_j\cos(\alpha) - Z_P\frac{\sin(\psi)\sin(\theta)^2}{\DD} +
    x_2(R)\rt)^2\\
\end{array}
$$



$$
\lt\{
	\begin{array}{lll}
		x_2(c) & = z_p\frac{\sin(\psi)\sin(\theta)^2}{\dd}\\
        a & = z_p\frac{\sin(\theta)\cos(\psi)}{\sqrt{\dd}}\\
		b & = z_p\frac{\sin(\theta)\cos(\psi)\cos(\theta)}{\dd}
	\end{array}
\rt.
$$

Hence:
$$
\lt\{
    \begin{array}{lll}
        \lt(x_i\cos(\alpha) - x_j\sin(\alpha) + x_1(R)\rt)\sqrt{\dd} = z_p\sin(\theta)\cos(\psi)\cos(\epsilon)\\
        \lt(x_i\sin(\alpha) + x_j\cos(\alpha) + x_2(R)\rt)\dd =
        z_p\sin(\theta)\lt(\sin(\psi)\sin(\theta) + \cos(\psi)\cos(\theta)\sin(\epsilon)\rt)
    \end{array}
\rt.
$$



\appendix
\chapter{Appendices}

\section{Section}
\subsection{Subsection}

\end{document}
