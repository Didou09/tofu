\documentclass[a4paper,11pt,twoside,titlepage,openright]{book}

\usepackage[english]{babel}
\usepackage{color}
\usepackage{graphicx}
\usepackage{amsmath}
\numberwithin{equation}{section}
\usepackage[margin=3cm]{geometry}
\usepackage{hyperref}
\usepackage{epsfig,amsfonts}
\usepackage{xcolor,import}


\pagestyle{plain}

\newcommand{\ud}[1]{\underline{#1}}
\newcommand{\lt}{\left}
\newcommand{\rt}{\right}
\DeclareMathOperator{\e0}{\epsilon_0}
\DeclareMathOperator{\n}{\underline{n}}
\DeclareMathOperator{\ei}{\underline{e}_1}
\DeclareMathOperator{\et}{\underline{e}_2}
\DeclareMathOperator{\ex}{\underline{e}_x}
\DeclareMathOperator{\ey}{\underline{e}_y}
\DeclareMathOperator{\ez}{\underline{e}_z}
\DeclareMathOperator{\DD}{\cos(\theta)^2 - \sin(\psi)^2}
\newcommand{\wdg}{\wedge}
\newcommand{\hypot}[1]{\textbf{\textcolor{green}{#1}}}


\begin{document}

\title{ToFu geometric tools\\ Intersection of a cone with a plane}
\author{Didier VEZINET}
\date{15.10.2019}
\maketitle

\tableofcontents

\chapter{Geometry}
\label{chap:Definitions}

\section{Generic cone and plane}

Let's consider a half-cone $C_1$ (defined only for $z > 0$), with summit on the cartesian frame's origin $(O, \ex, \ey, \ez)$.
The cone's axis is the $(O,\ez)$ axis.
It's angular opening is $\theta$.

Let's consider plane $P_1$, of normal $\n$, intersection axis $(O,\ez)$ at point $P$ of coordinates $(0,0,Z_P)$.
Vector $\n$ is oriented by angles $\phi$ and $\psi$ such that one can define the local frame $(P, \ei, \et, \n)$:
$$
\lt\{
	\begin{array}{ll}
		\ei & = \cos(\phi)\ex + \sin(\phi)\ey\\
		\et & = \lt(-\sin(\phi)\ex + \cos(\phi)\ey\rt)\cos(\psi) + \sin(\psi)\ez\\
		\n & = \ei \wdg \et\\
		   & = \lt( \sin(\phi)\ex - \cos(\phi)\ey \rt)\sin(\psi) + \cos(\psi)\ez
	\end{array}
\rt.
$$

We want to find all points $M$ of coordinates $(x, y, z)$ and $(x_1, x_2)$ belonging both to the cone $C_1$ and the plane $P_1$.

$$
M \in C_1 \Leftrightarrow \underline{OM}.\ez = \cos(\theta) \|\underline{OM}\|
$$

$$
M \in P_1 \Leftrightarrow \underline{PM}.\n = 0
$$


\section{Intersection}

If $M$ belongs to both $P_1$ and $C_1$, then:
$$
(\underline{OM}.\ez)^2 = \cos(\theta)^2 \|\underline{OM}\|^2
$$

Given that:
$$
\begin{array}{lll}
	\underline{OM} & = \underline{OP} + \underline{PM}\\
		       & = Z_P\ez + x_1\ei + x_2\et\\
		       & = Z_P\ez + x_1\lt(\cos(\phi)\ex + \sin(\phi)\ey\rt) + x_2\lt(\lt(-\sin(\phi)\ex + \cos(\phi)\ey\rt)\cos(\psi) + \sin(\psi)\ez\rt)\\
		       & = Z_P\ez + x_1\cos(\phi)\ex + x_1\sin(\phi)\ey - x_2\sin(\phi)\cos(\psi)\ex + x_2\cos(\phi)\cos(\psi)\ey + x_2\sin(\psi)\ez\\
	               & = \lt(x_1\cos(\phi) - x_2\sin(\phi)\cos(\psi)\rt)\ex + \lt(x_1\sin(\phi) + x_2\cos(\phi)\cos(\psi)\rt)\ey + \lt(Z_P + x_2\sin(\psi)\rt)\ez\\
\end{array}
$$

We have:
$$
\begin{array}{lll}
	(\underline{OM}.\ez)^2  & = \lt(Z_P + x_2\sin(\psi)\rt)^2\\
				& = Z_P^2 + 2Z_Px_2\sin(\psi) + x_2^2\sin(\psi)^2
\end{array}
$$

And:
$$
\begin{array}{lll}
	\|\underline{OM}\|^2 	& = & \| \lt(x_1\cos(\phi) - x_2\sin(\phi)\cos(\psi)\rt)\ex + \lt(x_1\sin(\phi) + x_2\cos(\phi)\cos(\psi)\rt)\ey + \lt(Z_P + x_2\sin(\psi)\rt)\ez\|^2\\
				& = & \lt( x_1\cos(\phi) - x_2\sin(\phi)\cos(\psi) \rt)^2\\
				&   & + \lt(x_1\sin(\phi) + x_2\cos(\phi)\cos(\psi)\rt)^2\\
				&   & + \lt(Z_P + x_2\sin(\psi)\rt)^2\\
				& = & x_1^2\cos(\phi)^2 - 2x_1x_2\cos(\phi)\sin(\phi)\cos(\psi) + x_2^2\sin(\phi)^2\cos(\psi)^2\\
				&   & + x_1^2\sin(\phi)^2 + 2x_1x_2\sin(\phi)\cos(\phi)\cos(\psi) + x_2^2\cos(\phi)^2\cos(\psi)^2\\
				&   & + Z_P^2  + 2Z_Px_2\sin(\psi) + x_2^2\sin(\psi)^2\\
				& = & x_1^2 + x_2^2\cos(\psi)^2\\
				&   & + Z_P^2  + 2Z_Px_2\sin(\psi) + x_2^2\sin(\psi)^2\\
				& = & x_1^2 + x_2^2 + 2Z_Px_2\sin(\psi) + Z_P^2\\
\end{array}
$$

Thus:
$$
\begin{array}{lll}
	& (\underline{OM}.\ez)^2 = \cos(\theta)^2 \|\underline{OM}\|^2\\
	\Leftrightarrow & Z_P^2 + 2Z_Px_2\sin(\psi) + x_2^2\sin(\psi)^2 = \cos(\theta)^2 \lt(x_1^2 + x_2^2 + 2Z_Px_2\sin(\psi) + Z_P^2\rt)\\
	\Leftrightarrow & Z_P^2\lt(1-\cos(\theta)^2\rt) + 2Z_Px_2\sin(\psi)\lt(1-\cos(\theta)^2\rt) = x_1^2\cos(\theta)^2 + x_2^2\lt(\cos(\theta)^2 - \sin(\psi)^2\rt)\\
	\Leftrightarrow & Z_P^2\sin(\theta)^2 + 2Z_Px_2\sin(\psi)\sin(\theta)^2 = x_1^2\cos(\theta)^2 + x_2^2\lt(\cos(\theta)^2 - \sin(\psi)^2\rt)\\
\end{array}
$$

Considering that by hypothesis $\theta>0$:
$$
\begin{array}{lll}
	& (\underline{OM}.\ez)^2 = \cos(\theta)^2 \|\underline{OM}\|^2\\
	\Leftrightarrow & x_1^2\cos(\theta)^2 + x_2^2\lt(\DD\rt) - 2Z_Px_2\sin(\psi)\sin(\theta)^2 - Z_P^2\sin(\theta)^2 = 0\\
	\Leftrightarrow & x_1^2\frac{\cos(\theta)^2}{\DD} + x_2^2 - 2x_2Z_P\frac{\sin(\psi)\sin(\theta)^2}{\DD} - Z_P^2\frac{\sin(\theta)^2}{\DD} = 0\\
	\Leftrightarrow & x_1^2\frac{\cos(\theta)^2}{\DD} + \lt(x_2 - Z_P\frac{\sin(\psi)\sin(\theta)^2}{\DD}\rt)^2 - Z_P^2\frac{\sin(\psi)^2\sin(\theta)^4}{\lt(\DD\rt)^2} - Z_P^2\frac{\sin(\theta)^2}{\DD} = 0\\
	\Leftrightarrow & x_1^2\frac{\cos(\theta)^2}{\DD} + \lt(x_2 - Z_P\frac{\sin(\psi)\sin(\theta)^2}{\DD}\rt)^2 = Z_P^2 \frac{\sin(\theta)^2}{\lt(\DD\rt)^2} \lt(\sin(\psi)^2\sin(\theta)^2 + \DD \rt)\\
	\Leftrightarrow & x_1^2\frac{\cos(\theta)^2}{\DD} + \lt(x_2 - Z_P\frac{\sin(\psi)\sin(\theta)^2}{\DD}\rt)^2 = Z_P^2 \frac{\sin(\theta)^2}{\lt(\DD\rt)^2} \lt(-\sin(\psi)^2\cos(\theta)^2 + \cos(\theta)^2 \rt)\\
	\Leftrightarrow & x_1^2\frac{\cos(\theta)^2}{\DD} + \lt(x_2 - Z_P\frac{\sin(\psi)\sin(\theta)^2}{\DD}\rt)^2 = Z_P^2 \frac{\sin(\theta)^2\cos(\psi)^2\cos(\theta)^2}{\lt(\DD\rt)^2} \\
	\Leftrightarrow & \frac{x_1^2}{ Z_P^2\frac{\sin(\theta)^2\cos(\psi)^2}{\DD} } + \frac{\lt(x_2-Z_P\frac{\sin(\psi)\sin(\theta)^2}{\DD}\rt)^2}{ Z_P^2\frac{\sin(\theta)^2\cos(\psi)^2\cos(\theta)^2}{\lt(\DD\rt)^2} } = 1
\end{array}
$$

Or, in a reduced conic form:

$$
\frac{x_1^2}{ a^2 } + \frac{\lt(x_2-x_2(C)\rt)^2}{b^2} = 1
$$


With:
$$
\lt\{
	\begin{array}{lll}
		x_2(C) & = Z_P\frac{\sin(\psi)\sin(\theta)^2}{\DD} & \text{    $x_2$ coordinate of the center} \\
		a^2 & = Z_P^2\frac{\sin(\theta)^2\cos(\psi)^2}{\DD} & \text{    squared minor radius} \\
		b^2 & = Z_P^2\frac{\sin(\theta)^2\cos(\psi)^2\cos(\theta)^2}{\lt(\DD\rt)^2} & \text{    squared major radius} \\
		b^2 & = a^2\frac{\cos(\theta)^2}{\DD} \Leftrightarrow a^2 = b^2\lt(1-\frac{\sin(\psi)^2}{\cos(\theta)^2}\rt)
	\end{array}
\rt.
$$

The distance $d_{CF}$ between the center $C$ and the focal point $F$ can be deduced from:
$$
\begin{array}{lll}
	d_{CF}^2 & = b^2-a^2\\
		& = b^2\frac{\sin(\psi)^2}{\cos(\theta)^2}\\
		& = Z_P^2\frac{\sin(\theta)^2\cos(\psi)^2\sin(\psi)^2}{\lt(\DD\rt)^2}
\end{array}
$$

Hence, the $x_2$ coordinate of $F$ is:
$$
\begin{array}{lll}
	x_2(F) & = x_2(C) \pm d_{CF}\\
	       & = Z_P\frac{\sin(\psi)\sin(\theta)^2}{\DD} \pm Z_P\frac{\sin(\theta)\cos(\psi)\sin(\psi)}{\DD}\\
	       & = Z_P\frac{\sin(\psi)\sin(\theta)^2 \pm \sin(\theta)\cos(\psi)\sin(\psi)}{\DD}\\
	       & = Z_P\frac{\sin(\psi)\sin(\theta)}{\DD}\lt(\sin(\theta) \pm \cos(\psi)\rt)\\
\end{array}
$$

It is worth noticing that the neither the focal point nor the center correspond to the intersection between the axes and the plane $P$.


\section{Parametric equation}

In our case, only the axes $(O, \ez)$, fixed by the crystal's summit and normal, is independent from the cone's angular opening $\theta$.
It makes sense to design an ad-hoc coordinate system centered on the ellipse's center $C$ to use its parameterized equation.

Knowing all geometrical parameters, it is possible to compute all points on the ellipse parameterizing them with angle $\epsilon$:
$$
\lt\{
	\begin{array}{lll}
		x_1 = a\cos(\epsilon)\\
		x_2 = x_2(C) + b\sin(\epsilon)
	\end{array}
\rt.
$$

Keep in mind that the frame $(P, \ei, \et)$ is, by definition ligned on the minor and major axes of the ellipse.
Hence, for an arbitrary frame $(R, \ud{e}_i, \ud{e}_j)$ on plane $P_1$, translated and rotated by $\alpha$ with respect to $(P, \ei, \et)$:

$$
\lt\{
	\begin{array}{lll}
		\ud{e}_i = \cos(\alpha)\ei + \sin(\alpha)\et\\
		\ud{e}_j = -\sin(\alpha)\ei + \cos(\alpha)\et\\
		\ei = \cos(\alpha)\ud{e_i} - \sin(\alpha)\ud{e}_j
		\et = \sin(\alpha)\ud{e_i} + \cos(\alpha)\ud{e}_j
	\end{array}
\rt.
$$

Or, in coordinate tranforms:
$$
\lt\{
	\begin{array}{lll}
		x_1 = x_1(R) + x_i\cos(\alpha) - x_j\sin(\alpha)\\
		x_2 = x_2(R) + x_i\sin(\alpha) + x_j\cos(\alpha)\\
		x_i = (x_1-x_1(R))\cos(\alpha) + (x_2-x_2(R))\sin(\alpha)\\
		x_j = -(x_1-x_1(R))\sin(\alpha) + (x_2-x_2(R))\cos(\alpha)
	\end{array}
\rt.
$$

Hence:
$$
\lt\{
	\begin{array}{lll}
		x_i = \lt(a\cos(\epsilon)-x_1(R)\rt)\cos(\alpha) + \lt(x_2(C)-x_2(R) + b\sin(\epsilon)\rt)\sin(\alpha)\\
		x_j = -\lt(a\cos(\epsilon)-x_1(R)\rt)\sin(\alpha) + \lt(x_2(C)-x_2(R) + b\sin(\epsilon)\rt)\cos(\alpha)
	\end{array}
\rt.
$$










%\section{Coordinate Transforms}

%Let's consider the same cone and plane as before.
%But this time we are looking at a small fraction of the ellipse, the fraction intersecting a bounded box on the plane, with a local frame $(R, \ud{e}_i, \ud{e}_j)$, rotated by an angle $\ksi$ from the natural plane frame such that:
%$$
%\lt\{
	%\begin{array}{lll}
		%\ud{e}_i = \cos(\ksi)\ei + \sin(\ksi)\et\\
		%\ud{e}_j = -\sin(\ksi)\ei + \cos(\ksi)\et\\
	%\end{array}
%\rt.
%$$

%And reciprocately:
%$$
%\lt\{
	%\begin{array}{lll}
		%\ei = \cos(\ksi)\ud{e}_i - \sin(\ksi)\ud{e}_j\\
		%\et = \sin(\ksi)\ud{e}_i + \cos(\ksi)\ud{e}_j\\
	%\end{array}
%\rt.
%$$

%Such that:
%$$
%\begin{array}{lll}
	%\ud{CM} & = \ud{CR} + \ud{RM}\\
		%& = x_1(R)\ei + X_2(R)\et + i\ud{e}_i + j\ud{e}_j\\
		%& = \lt(x_1(R) + i\cos(\ksi) - j\sin(\ksi)\rt)\ei + \lt(X_2(R) + i\sin(\ksi) + j\cos(\ksi)\rt)\et\\
%\end{array}
%$$

%Keeping in mind that:
%$$
%X_2 = x_2 - Z_F\sin(\theta)^2 \frac{\sin(\psi)}{\DD}
%$$

%We can write:
%$$
%x_1 = x_1(R) + i\cos(\ksi) - j\sin(\ksi)
%X_2 = x_2(R) - Z_F\sin(\theta)^2 \frac{\sin(\psi)}{\DD} + i\sin(\ksi) + j\cos(\ksi)
%$$

%We want to easily transform between $(i,j)$ and $(\theta, \espilon)$, where $\epsilon$ is the angle of a point along the ellipse of cone width $\theta$, with respect to its center $C$.


%\subsection{From cone to local cartesian}

%Knowing all geometrical parameters, we can infer:
%$$
%\begin{array}{lll}
	%\underline{RM}  & = i\ud{e}_i + j\ud{e}_j\\
			%& = \underline{RC} + \underline{CM}\\
			%& = -x_1(R)\ei - X_2(R)\et + x_1\ei + X_2\et\\
%\end{array}
%$$




%\subsection{From local cartesian to cone}

%$$
%x1 =
%$$



\appendix
\chapter{Acceleration radiation from a unique point-like charge}

\section{Retarded time and potential}
\subsection{Retarded time}


Hence $\frac{dR(t_r)}{c} + dt_r = dt$


\end{document}
