\documentclass[a4paper,11pt,twoside,titlepage,openright]{book}

\usepackage[english]{babel}
\usepackage{color}
\usepackage{graphicx}
\usepackage{amsmath}
\numberwithin{equation}{section}
\usepackage[margin=3cm]{geometry}
\usepackage{hyperref}
\usepackage{epsfig,amsfonts}
\usepackage{xcolor,import}


\pagestyle{plain}

\newcommand{\ud}[1]{\underline{#1}}
\newcommand{\e}[1]{\underline{e}_{#1}}
\newcommand{\lt}{\left}
\newcommand{\rt}{\right}
\newcommand{\lra}{\Leftrightarrow}
\DeclareMathOperator{\n}{\underline{n}}
\DeclareMathOperator{\ex}{\underline{e}_x}
\DeclareMathOperator{\ey}{\underline{e}_y}
\DeclareMathOperator{\ez}{\underline{e}_z}
\DeclareMathOperator{\bragg}{\theta_{bragg}}
\DeclareMathOperator{\Z}{Z_C}
\DeclareMathOperator{\DD}{\cos(\theta)^2 - \sin(\psi)^2}
\newcommand{\wdg}{\wedge}
\newcommand{\hypot}[1]{\textbf{\textcolor{green}{#1}}}


\begin{document}

\title{ToFu geometric tools\\ Intersection of a cone with a circle (magnetic axis)}
\author{Didier VEZINET}
\date{04.12.2019}
\maketitle

\tableofcontents

\chapter{Geometry}
\label{chap:Geometry}

\section{Generic cone and plane}

Let's consider a cartesian frame $(O, \ex, \ey, \ez)$.
Let's consider a half-cone defined by its axis $(S, \n)$ and half-opening $\alpha = \pi/2 - \bragg$.
The coordinates of $S$ are $(x_S, y_S, z_S)$.
The coordinates of $\n$ are $(n_x, n_y, n_z)$.
Let's consider a circle of axis $(O, \ez)$, of radius $R$, centered on $C$ or coordinates $(0, 0, \Z)$.

Let's consider point $M$ of coordinates $(x, y, z)$ and $(R, \theta, z)$ belonging to both the cone and the circle.


$$
\lt\{
	\begin{array}{ll}
		M \in circle & \lra \ud{OM} = \Z\ez + R(\cos(\theta)\ex + \sin(\theta)\ey)\\
		M \in cone & \lra \ud{SM}.\n = \cos(\alpha)\|\ud{SM}\|
	\end{array}
\rt.
$$



\section{Intersection}

If $M$ belongs to both the circle and the cone, then:

$$
\begin{array}{ll}
	& \lt[\lt(\ud{SO}+\ud{OM}\rt).\n\rt]^2 = \cos^2(\alpha)\|\ud{SO} + \ud{OM}\|^2\\
	\lra &
	\lt(\ud{SO}.\n\rt)^2 + \lt(\ud{OM}.\n\rt)^2 + 2\lt(\ud{SO}.\n\rt)\lt(\ud{OM}.\n\rt)
	= \cos^2(\alpha)\lt[\|\ud{SO}\|^2 + \|\ud{OM}\|^2 + 2\ud{SO}.\ud{OM}\rt]\\
	\lra &
	\lt(\ud{OM}.\n\rt)^2 + 2\lt(\ud{SO}.\n\rt)\lt(\ud{OM}.\n\rt)
	- \cos^2(\alpha)\|\ud{OM}\|^2 - 2\cos^2(\alpha)\ud{SO}.\ud{OM} + A = 0
\end{array}
$$

Where we have introduced $A = \lt(\ud{SO}.\n\rt)^2 - \cos^2(\alpha)\|\ud{SO}\|^2$

Now, we can write:
$$
\lt\{
	\begin{array}{lll}
		\|\ud{OM}\|^2 & = & \Z^2 + R^2\\
		\ud{OM}.\n & = & \Z n_z + R\cos(\theta)n_x + R\sin(\theta)n_y\\
		\lt(\ud{OM}.\n\rt)^2 & = & \lt(\Z n_z\rt)^2 + \lt(R\cos(\theta)n_x\rt)^2 + \lt(R\sin(\theta)n_y\rt)^2\\
				     &&  + 2\Z R\cos(\theta)n_xn_Z + 2\Z R\sin(\theta)n_yn_z + 2R^2\cos(\theta)\sin(\theta)n_xn_y\\
		\ud{SO}.\ud{OM} & = & -\Z z_S - Rx_S\cos(\theta) - Ry_S\sin(\theta)
	\end{array}
\rt.
$$

Hence:
$$
\begin{array}{ll}
	& \lt(\ud{OM}.\n\rt)^2 + 2\lt(\ud{SO}.\n\rt)\lt(\ud{OM}.\n\rt)
	- \cos^2(\alpha)\|\ud{OM}\|^2 - 2\cos^2(\alpha)\ud{SO}.\ud{OM} + A\\
	= &
	\lt(\Z n_z\rt)^2 + \lt(R\cos(\theta)n_x\rt)^2 + \lt(R\sin(\theta)n_y\rt)^2\\
	& + 2\Z R\cos(\theta)n_xn_Z + 2\Z R\sin(\theta)n_yn_z + 2R^2\cos(\theta)\sin(\theta)n_xn_y\\
	& +2\lt(\ud{SO}.\n\rt)\lt(\Z n_z + R\cos(\theta)n_x + R\sin(\theta)n_y\rt)\\
	& - \cos^2(\alpha)\Z^2 - \cos^2(\alpha)R^2\\
	& + 2\cos^2(\alpha)\lt(\Z z_S + Rx_S\cos(\theta) + Ry_S\sin(\theta)\rt) + A
\end{array}
$$







\section{Parametric equation}

\subsection{From bragg angle and parameter to local cartesian coordinates}







\appendix
\chapter{Appendices}

\section{Section}
\subsection{Subsection}

\end{document}
