\documentclass[a4paper,11pt,twoside,titlepage,openright]{book}

\usepackage[english]{babel}
\usepackage{color}
\usepackage{graphicx}
\usepackage{amsmath}
\numberwithin{equation}{section}
\usepackage[margin=3cm]{geometry}
\usepackage{hyperref}
\usepackage{epsfig,amsfonts}
\usepackage{xcolor,import}


\pagestyle{plain}

\newcommand{\ud}[1]{\underline{#1}}
\newcommand{\lt}{\left}
\newcommand{\rt}{\right}
\DeclareMathOperator{\e0}{\epsilon_0}
\newcommand{\wdg}{\wedge}
\newcommand{\emis}{\emph{emissivity}}
\newcommand{\ema}{\epsilon^{\eta}}
\newcommand{\hypot}[1]{\textbf{\textcolor{green}{#1}}}


\begin{document}

\title{ToFu geometric tools\\ Intersection of a LOS with a cone}
\author{Didier VEZINET \and Laura S. Mendoza}
\date{02.06.2017}
\maketitle

\tableofcontents




\chapter{Definitions}

\section{Geometry definition in ToFu}

The definition of a fusion device in ToFu is done by defining the edge of a poloidal plane as a set of segments in a 2D plane. The 3D volume is obtained by an extrusion for cylinders or a revolution for tori.
We consider an orthonormal direct cylindrical coordinate system $(O,\ud{e}_R,\ud{e}_{\theta},\ud{e}_Z)$ associated to the orthonormal direct cartesian coordinate system $(O,\ud{e}_X,\ud{e}_Y,\ud{e}_Z)$. We suppose that all poloidal planes live in $(R,Z)$ and can be obtained after a revolution around the $Z$ axis of the user-defined poloidal plane at $\theta=0$, $\mathcal{P}_0$. Thus, the torus is axisymmetric around the $(O,Z)$ axis (see Figure~\ref{fig:tok-ab}).




\begin{figure}[h]
\centering{
\def\svgwidth{0.75\linewidth}
\import{figures/}{tore_cones12.pdf_tex} 
\caption{Two examples of a circular torus approximated by a revolved octagon. For each segment $\overline{AB}$ of the octagon there is a cone with origin on the $(O,Z)$ axis.}
\label{fig:tok-ab}
}
\end{figure}

\section{Notations}

In order to simplify the computations, let $A$ and $B$ be the end points of a segment $\mathcal{S}_i$ such that $A\neq B$ and $\mathcal{P}_0 = \cup_{i=1}^{n} \mathcal{S}_i = \cup_{i=1}^n \overline{{\rm A}_i{\rm B}_i}$ with $n$ the number of segments given by the user defining the plane $\mathcal{P}_0$. We define a right circular cone $\mathcal{C}$ of origin $P = ({\rm A},{\rm B}) \cap ({\rm O}, {\rm Z})$ of generatrix $(A,B)$ and of axis $(O,Z)$ (see Figure~\ref{fig:tok-ab}). Thus we can define the edge of the torus as the union of the edges of the frustums $\mathcal{F}_i$ defined by truncating the cones $\mathcal{C}_i$ to the segment $\overline{AB}_i$.



Then, any point $M$ with coordinates $(X,Y,Z)$ or $(R,\theta,Z)$ belongs to the frustum $\mathcal{F}$ if and only if
$$
\exists q \in [0;1] / 
\left\{ \begin{array}{ll}
R-R_A = q(R_B-R_A)\\
Z-Z_A = q(Z_B-Z_A)
\end{array}\right.
$$


Now let us consider a LOS $L$ (i.e.: a half-infinite line) defined by a point $D$ and a normalized directing vector $u$, of respective coordinates $(X_D,Y_D,Z_D)$ or $(R_D,\theta_D,Z_D)$ and $(u_X,u_Y,u_Z)$.
Then, point M belongs to $L$ if and only if:
$$
\exists k \in [0;\infty[ / \ud{DM} = k\ud{u}
$$


\chapter{Derivation}

Let us now consider all intersections between the edge of a frustum $\mathcal{F}$ and a semi-line $L$.

\begin{equation}
\begin{array}{lll}
\exists (q,k) \in [0;1]\times [0;\infty[ /& & 
\left\{\begin{array}{ll}
R-R_A = q(R_B-R_A)\\
Z-Z_A = q(Z_B-Z_A)\\
X-X_D = ku_X\\
Y-Y_D = ku_Y\\
Z-Z_D = ku_Z
\end{array}\right.\\
\end{array}
\label{eqn:201}
\end{equation}

Which yields (by combining to keep only unknowns $q$ and $k$):

\begin{equation}
\begin{array}{cc}
q(Z_B-Z_A) = Z_D-Z_A + ku_Z\\
q^2(R_B-R_A)^2 + 2qR_A(R_B-R_A) = \left(k\ud{u}_{//} + \ud{D}_{//}\right)^2 - R_A^2
\end{array}
\end{equation}

Where we have introduced $R_D = \sqrt{X_D^2+Y_D^2}$, $\ud{u}_{//} = u_X\ud{e}_X+u_Y\ud{e}_Y$ and $\ud{D}_{//} = X_D\ud{e}_X + Y_D\ud{e}_Y$.
We can then derive a decision tree.


Given that the parallelization will take place on the LOS (i.e.: not on the cones which are parts of the vacuum vessel), we will discriminate case based prioritarily on the components of $\ud{u}$ and $D$.
We will detail only the cases which have solutions, in order to make it as clear as possible for implementation of an efficient algorithm.
We will also only consider non-tangential solution, as we are looking for entry/exit points.

\section{Horizontal LOS: $u_Z=0$}

Let us consider an horizontal LOS, such that $u_Z=0$, then \eqref{eqn:201} becomes

$$
\begin{array}{lll}
\exists (q,k) \in [0;1]\times [0;\infty[ /& &
\left\{\begin{array}{ll}
R-R_A = q(R_B-R_A)\\
Z_D-Z_A = q(Z_B-Z_A)\\
X-X_D = ku_X\\
Y-Y_D = ku_Y\\
Z=Z_D
\end{array}\right.\\
\end{array}
$$

From here we can differentiate two cases regarding the frustum $\mathcal{F}$.

\subsection{Plane Frustum: $Z_B = Z_A$}

Let us consider first the case where $Z_B=Z_A$, when the frustum becomes an annulus on the $(X,Y)$ plane, then we will have two different cases.

\begin{itemize}
\item $Z_D\neq Z_A \Rightarrow$ the cone and the LOS stand in different parallel planes $\Rightarrow$ no solution.
\item $Z_D=Z_A \Rightarrow$ the cone stands in the same plane as the LOS (see \ref{fig:hoz-frus-hoz-los}) $\Rightarrow$ infinity of solutions, we consider no solutions as this is a limit case with no clearly identified intersection.
\end{itemize}

\begin{figure}[h]
\centering{
\def\svgwidth{0.5\linewidth}
\import{figures/}{horizontal_cone_hoz_LOS.pdf_tex} 
\caption{Plane frustum and horizontal Line of Sight on the same $Z$-plane.}
\label{fig:hoz-frus-hoz-los}
}
\end{figure}

Hence, the only derivable solutions suppose that $Z_B\neq Z_A$.

\subsection{Non-horizontal cone: $Z_B\neq Z_A$}

Then $q=\dfrac{Z_D-Z_A}{Z_B-Z_A}$.
There are acceptable solution only if $q\in[0;1]$. 
By introducing $$C = q^2(R_B-R_A)^2 + 2qR_A(R_B-R_A) + R_A^2,$$ we have
$$
\left(k\ud{u}_{//} + \ud{D}_{//}\right)^2 - C = 0\\
\Leftrightarrow k^2\ud{u}_{//}^2 + 2k\ud{u}_{//}\cdot\ud{D}_{//} + \ud{D}_{//}^2-C = 0
$$

Then introducing $\Delta = 4\left(\ud{u}_{//}\cdot\ud{D}_{//}\right)^2 - 4\ud{u}_{//}^2\left(\ud{D}_{//}^2-C\right) = 4\delta$, there are non-tangential solutions only if $\left(\ud{u}_{//}\cdot\ud{D}_{//}\right)^2 >\ud{u}_{//}^2\left(\ud{D}_{//}^2-C\right)$.
It is necessary to compute the solutions k because we need to check if $k>=0$.

$$
k_{1,2} = \dfrac{-\ud{u}_{//}\cdot\ud{D}_{//} \pm \sqrt{\delta}}{\ud{u}_{//}^2}
$$

Hence, we have solutions if:
$$
\left\{
\begin{array}{lll}
u_Z = 0\\
Z_B\neq Z_A\\
\dfrac{Z_D-Z_A}{Z_B-Z_A} \in [0;1]\\
k_{1,2} = \dfrac{-\ud{u}_{//}\cdot\ud{D}_{//} \pm \sqrt{\delta}}{\ud{u}_{//}^2} \geq 0
\end{array}
\right.
$$

\section{Non-horizontal LOS: $u_Z\neq0$}

Then $k=q\dfrac{Z_B-Z_A}{u_Z} - \dfrac{Z_D-Z_A}{u_Z}$, which means:


$$
\begin{array}{lll}
q^2 & (R_B-R_A)^2 + 2qR_A(R_B-R_A) + R_A^2\\
& =\left(\left(q\dfrac{Z_B-Z_A}{u_Z} - \dfrac{Z_D-Z_A}{u_Z}\right)\ud{u}_{//} + \ud{D}_{//}\right)^2\\
& = \left(q\dfrac{Z_B-Z_A}{u_Z} - \dfrac{Z_D-Z_A}{u_Z}\right)^2\ud{u}_{//}^2 + 2\left(q\dfrac{Z_B-Z_A}{u_Z} - \dfrac{Z_D-Z_A}{u_Z}\right)\ud{u}_{//}\cdot\ud{D}_{//} + \ud{D}_{//}^2\\
& =q^2\left(\dfrac{Z_B-Z_A}{u_Z}\right)^2\ud{u}_{//}^2 - 2q\dfrac{Z_B-Z_A}{u_Z}\dfrac{Z_D-Z_A}{u_Z}\ud{u}_{//}^2 \\
& + \left(\dfrac{Z_D-Z_A}{u_Z}\right)^2\ud{u}_{//}^2
+ 2q\dfrac{Z_B-Z_A}{u_Z}\ud{u}_{//}\cdot\ud{D}_{//} - 2\dfrac{Z_D-Z_A}{u_Z}\ud{u}_{//}\cdot\ud{D}_{//} + \ud{D}_{//}^2\\ 
\end{array}
$$

Hence:
$$
\begin{array}{lll}
0 &=& q^2\left( (R_B-R_A)^2 - \left(\frac{Z_B-Z_A}{u_Z}\right)^2\ud{u}_{//}^2 \right)\\
& + & 2q\left( R_A(R_B-R_A) + \frac{Z_B-Z_A}{u_Z}\frac{Z_D-Z_A}{u_Z}\ud{u}_{//}^2 - \frac{Z_B-Z_A}{u_Z}\ud{u}_{//}\cdot\ud{D}_{//}  \right)\\
& - & \left(\frac{Z_D-Z_A}{u_Z}\right)^2\ud{u}_{//}^2 + 2\frac{Z_D-Z_A}{u_Z}\ud{u}_{//}\cdot\ud{D}_{//} - \ud{D}_{//}^2 + R_A^2 
\end{array}
$$

We can then introduce:
$$
\left\{
\begin{array}{ll}
A = (R_B-R_A)^2 - \left(\frac{Z_B-Z_A}{u_Z}\right)^2\ud{u}_{//}^2\\
B = R_A(R_B-R_A) + \frac{Z_B-Z_A}{u_Z}\frac{Z_D-Z_A}{u_Z}\ud{u}_{//}^2 - \frac{Z_B-Z_A}{u_Z}\ud{u}_{//}\cdot\ud{D}_{//}\\
C = -\left(\frac{Z_D-Z_A}{u_Z}\right)^2\ud{u}_{//}^2 + 2\frac{Z_D-Z_A}{u_Z}\ud{u}_{//}\cdot\ud{D}_{//} - \ud{D}_{//}^2 + R_A^2
\end{array}
\right.
$$

Because of the shape of potential solutions, we have to discriminate the case $A=0$.

\subsection{$A=0$: LOS parallel to one of the cone generatrices}

Then, because of the shape of the potential solution, we have to discriminate the case $B=0$.
But in this case we have $C=0$. 
\begin{itemize}
\item if $C=0 \Rightarrow$ no condition on q and k, the LOS is included in the cone $\Rightarrow$ we consider no solution
\item if $C\neq 0 \Rightarrow$ Impossible, no solution
\end{itemize}

Only the case $B\neq 0$ is thus relevant.

\subsubsection{$B\neq0$: LOS not included in the cone}

Then, there is either one or no solution:
$$
\left\{
\begin{array}{ll}
q = -\frac{C}{2B} & \in [0,1]\\
k = q\frac{Z_B-Z_A}{u_Z} - \frac{Z_D-Z_A}{u_Z} & \geq 0
\end{array}
\right.
$$

\subsection{$A\neq 0$: LOS not parallel to a cone generatrix}

Then, we only consider cases with two distinct solutions (i.e.: no tangential case):
$$
\left\{
\begin{array}{ll}
B^2 > AC\\
q = \frac{-B \pm \sqrt{B^2-AC}}{A} & \in [0,1]\\
k = q\frac{Z_B-Z_A}{u_Z} - \frac{Z_D-Z_A}{u_Z} & \geq 0
\end{array}
\right.
$$





\appendix
\chapter{Acceleration radiation from a unique point-like charge}

\section{Retarded time and potential}
\subsection{Retarded time}

\subsubsection{Deriving the retarded time}
\label{Ap:RetardTime}

Hence $\frac{dR(t_r)}{c} + dt_r = dt$



\subsection{Retarded potentials}
\subsubsection{Deriving the potential propagation equations}
\label{Ap:PotentialPropagation}



\end{document}


\begin{figure}
  \caption{A picture of a gull.}
  \centering
    \includegraphics[width=0.5\textwidth]{gull}
\end{figure}


